% ----
% COMP1204 CW1 Report Document
% ----
\documentclass[]{article}
\usepackage{courier}
\usepackage{minted}
\usepackage{graphicx}

\usepackage[margin=1in]{geometry}
\setlength\parindent{0pt}

\title{COMP1204: Data Management \\ Coursework One: Hurricane Monitoring }
\author{student name: Zhenyang,Qin \\ student id: 34259686}

\begin{document}
\maketitle

\section{Introduction}
This work aims to extract useful storm data from a set of hurricane reports. The tag structure in these report is similar to XML file. Without using an XML parser,the data of a single storm is filtered and grouped by using regular expressions and Unix. Export multiple reports that are originally lengthy and contain many additional information into a file with a data format of CSV.This file only contain important information.

{\parskip=7pt
According to the time and coordinates provided in the data,the moving path of the storm can be drawn,and the storm plots can be obtained.The extracted data is exported in CSV format. Storm plots are generated as PNG files
from the CSV output using the provided create\_map\_plot.sh script.}

\section{Create CSV Script}

\begin{minted}[linenos]{bash}
#!/bin/bash

# 1 Set 2 varialbes to extract data from KML,and save as CSV

kml_input=$1
csv_output=$2

#   Set the first line of output file
echo "Timestamp,Latitude,Longitude,MinSeaLevelPressure,MaxIntensity" > $2

# 2 Collect the 5 required data
# 	Cut off useless parts then add unit at the end of each line

#   grep timestamp
grep "<dtg>" $1 | cut -d ">" -f 2 | cut -d "<" -f 1 > ts.txt

#   grep latitude
grep "<lat>" $1 | cut -d ">" -f 2 | cut -d "<" -f 1 \
    |  sed "s/$/& N/g" >lat.txt

#   grep longitude
grep "<lon>" $1 | cut -d ">" -f 2 | cut -d "<" -f 1 \
    | sed "s/$/& W/g" > lon.txt

#   grep max intensity
grep "<intensity>" $1 | cut -d ">" -f 2 | cut -d "<" -f 1 \
    | sed "s/$/& knots/g" > inten.txt

#   grep min sealevel pressure
grep "<minSea.*>" $1 | cut -d ">" -f 2 | cut -d "<" -f 1 \
    | sed "s/$/& mb/g" > press.txt

# 3 Paste data in order, output final data as a CSV file
paste -d ',' ts.txt lat.txt lon.txt press.txt inten.txt >>$2
\end{minted}

\section{Storm Plots}
\begin{figure}[ht]
    \centering
    \includegraphics[scale=0.67]{storm plot1.jpg}
    \caption{}
\end{figure}

\begin{figure}[ht]
    \centering
    \includegraphics[scale=0.5]{storm plot2.jpg}
    \caption{}
\end{figure}

\begin{figure}[ht]
    \centering
    \includegraphics[scale=0.5]{storm plot3.jpg}
    \caption{}
\end{figure}
\section{Git}
\end{document}
